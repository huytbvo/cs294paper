\section{Related Works}
\label{sec:related-work}

Our paper shares the same high level goal as
Halide~\cite{halide:siggraph}: decoupling the algorithm from the
scheduling to simplify the algorithm specification. Halide allows
programmers to write high performance image processing pipelines
without sacrificing readability in a domain-specific language that can
be compiled to different backends. Our pipeline synthesis tool allows
the user to describe a hardware datapath (the algorithm) and the pipeline
specification (the scheduling) for that datapath separately. The
Spiral~\cite{hoe:spiral} hardware generation framework and system
also follows a similar theme. Spiral allows users to generate hardware
for linear transforms from input problem specifications and directives
that describe the datapath. Nurvitadhi et al~\cite{hoe:syn} present
separate tools T-spec for transactional datapath specification and
T-piper for automatic pipeline synthesis. T-spec is used to describe
transactional datapaths as state elements and acyclic next-state logic
blocks that updates those state elements. Users manually annotate all
the state elements and next-stage logic blocks with the pipeline
stage number. T-piper analyzes the T-spec design to identify RAW
hazards and generate hazard resolution logic. Our tool is
different in that the datapath and pipeline specification are both
within the same language. The datapath specification does not have any
notion of stage number and the pipeline specification is maintained
separately from the datapath. Our synthesis tool then takes as input
the datapath and pipeline specification to infer pipeline register
placement and generate the appropriate hazard resolution logic.

