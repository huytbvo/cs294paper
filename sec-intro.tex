\section{Introduction}
There have been great productivity gains in digital design from using hardware description languages such as Verilog and VHDL to specify digital circuits. Further productivity gains be achieved by increasing the abstraction level in which digital designers describe hardware. The approach we take in this paper is to separate the description of data flow within a datapath from the the description of how to pipeline that datapath and automatically synthesize the pipelined datapath. Manually pipelining a datapath is timing consuming and error prone because the designer has to insert non-trivial control logic that becomes increasingly complex with increased pipeline depth. By separating the specification of the data flow from the specification of the pipelining, we allow the designer to more easily explore the design space of pipeline depth and pipeline hazard resolution options. 

In this paper we leverage Chisel's existing functionality as a wiring language to specify the data flow within a datapath, which consists of specifying a set of architectural state elements and their next state logic, and extend Chisel to allow for the separate specification of pipelining options, which includes pipeline depth and how to resolve each pipeline harzard that is present in the design, and the automatic synthesis of the pipelined datapath. We support resolving each pipeline hazard with one of three options: interlocking, bypassing, and speculation.


\cite{Bachrach-2012}
