\section{Introduction}
{\bf {\it Abstract- }} {\bf We augment Chisel to allow for the specification of a datapath to be separated from the specification of how to pipeline that datapath. The pipeline specification allows for the designer to specify the pipeline depth, placement of pipeline registers, and how to resolve pipline hazards. We created an automatic tool that combines the two specifications and generates a pipelined implementation of the datapath by performing transformations on the the Chisel Node graph.}

There have been great productivity gains in digital design through the use of relatively high level hardware description languages such as Verilog and VHDL to specify digital circuits. Additional productivity gains be achieved by further increasing the abstraction level in which digital designers describe hardware. The approach we take in this paper is to separate the description of data flow within a datapath, which we will now refer to as datapath specification, from the the description of how to pipeline that datapath, which we will now refer to as pipelining specification, and then automatically synthesize the pipelined datapath. Manually pipelining a datapath is timing consuming and error prone because the designer has to insert non-trivial control logic that becomes increasingly complex with increased pipeline depth. By separating the specification of the data flow from the specification of the pipelining and automatically synthesizing the pipelined datapath, we allow the designer to more easily explore the design space of pipeline depth and pipeline hazard resolution options. 

In this paper we leverage Chisel's existing functionality as a wiring language to specify the datapath and extend Chisel syntax to allow for a separate pipelining specification. We then use Chisel's Elaboration Time Interface to inject code that performs the automatic synthesis of the pipelined datapath from the separate datapath specification and pipelining specification.

